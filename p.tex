
\title{Controlling URL accesses in the enterprise by means of user-focused categorical classifiers}
% No se deben usar acr�nimos en el t�tulo - JJ
% Antonio - He puesto un t�tulo m�s serio y acorde con lo que se pretende hacer en el trabajo. El acr�nimo de URL no lo vamos a desplegar, digo yo. Si quer�is poned Web, pero vamos, que yo no lo veo mal as�.
%%%%%%%%%%%%%%%%%%%%%%%%%%%%%%%   AUTHORS   %%%%%%%%%%%%%%%%%%%%%%%%%%%%%%%

\author{P. de las Cuevas, A.M. Mora, J.J. Merelo}
\ead{\{paloma, amorag, jmerelo\}@geneura.ugr.es}
\address{Departamento de Arquitectura y Tecnolog�a de Computadores.\\ ETSIIT - CITIC. University of Granada, Spain}
%\author{A. M. Mora}
%\ead{amorag@geneura.ugr.es}
%\address{Departamento de Arquitectura y Tecnolog�a de Computadores. Escuela T�cnica Superior de Ingenier�as Inform�tica y de Telecomunicaci�n. CITIC. University of Granada, Spain}

%\maketitle

%
%%%%%%%%%%%%%%%%%%%%%%%%%%%%%%%%%   ABSTRACT   %%%%%%%%%%%%%%%%%%%%%%%%%%%%%%%%%
%
\begin{abstract} 

Many companies, concerned about the safety of the connections their employees establish to different web sites, implement different security techniques. 
One of the most extended involves use of blacklists and whitelists which, respectively, forbid or allow the access to the URLs included on them.
Their main problem is the difficulty for maintaining them up to date, along with the low flexibility they offer.
In this paper it is proposed a methodology for classifying unknown URLs as allowed or denied, based in previous accesses from the users to other sites.
This has been done using categorical classifiers, also taking into account the user influence in the accesses. Thus, a subset of the considered features have been selected based in some criteria, such as their influence on the classifiers' obtained rules and on their dependence on the user's actions.
This has been tested using real data gathered in an actual Spanish company, both at the access request and at the session level.
The obtained results show that \textbf{COMENTAR LOS RESULTADOS OBTENIDOS}. %revisar que el resumen sea correcto (objetivos) y completar comentando en una frase los resultados obtenidos.  [Pedro]
% Antonio - he intentado describir mejor lo que se hace (o se quiere hacer) en el trabajo.

\end{abstract}

